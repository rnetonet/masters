\documentclass[10pt]{beamer}

\usetheme{metropolis}
\usepackage{appendixnumberbeamer}

\usepackage{booktabs}
\usepackage{multirow}
\usepackage[scale=2]{ccicons}

\usepackage{pgfplots}
\usepgfplotslibrary{dateplot}

\usepackage{xspace}
\newcommand{\themename}{\textbf{\textsc{metropolis}}\xspace}
\usepackage{tabularx}

\usepackage[utf8]{inputenc}
\usepackage[brazilian]{babel}

\usepackage{blindtext}
\setbeamercolor{background canvas}{bg=white}

\title{}
\subtitle{Uso de Redes de Função de Base Radial e Cadeias de Markov para detecção online de mudanças de conceito em fluxos contínuos de dados}
\date{}
\author{\textbf{Discente:} Ruivaldo Neto \newline \textbf{Orientador:} Ricardo Rios}
\institute{Universidade Federal da Bahia \newline Departamento de Ciência da Computação \newline Programa de Pós-Graduação em Ciência da Computação \newline\newline Contato: rneto@rneto.dev \newline\newline 16 de Dezembro de 2019}

\titlegraphic{%
  \begin{picture}(0,0)
    \put(330, 28){\makebox(0,0)[rt]{\includegraphics[scale=0.25]{logo.png}}}
  \end{picture}
}

\begin{document}

\maketitle

\begin{frame}{Roteiro}
  \setbeamertemplate{section in toc}[sections numbered]
  \begin{minipage}{\textwidth}
    \tableofcontents
  \end{minipage}
\end{frame}

\section{Introdução}

\begin{frame}{Introdução}
    \begin{itemize}
        \item<1 -> Avanços tecnológicos recentes contribuiram para um aumento exponencial no volume de dados produzidos por sistemas computacionais \cite{idc_report}.
        \item<2 -> Parte significativa dos dados é produzida atráves de \alert{Fluxos Contínuos de Dados (FCDs)}: sequências \alert{ininterruptas} e \alert{potencialmente infinitas} de eventos \cite{Aggarwal:2006:DSM:1196418}.
        \item<3 -> FCDs estão presentes em diversos domínios de aplicação:
        \begin{itemize}
            \item Monitoramento de tráfico;
            \item Gestão de redes de telecomunicação;
            \item Análise do Mercado Financeiro;
            \item Detecção de intrusos.
        \end{itemize}
      \end{itemize}
\end{frame}

\begin{frame}{Introdução}
    \begin{itemize}
        \item<1 -> Técnicas de \alert{Aprendizado de Máquina (AM)} têm sido aplicadas para extrair informações úteis de grandes conjuntos de dados.
        \item<2 -> Cenários com FCDs limitam a aplicação de técnicas de AM, pois impõem restrições de tempo de resposta, de uso dos recursos computacionais e apresentam comportamento \alert{não estacionário}.
        \item<3 -> Em cenários não estacionários, o contexto do processo gerador e/ou a distribuição dos dados podem sofrer alterações (\alert{mudanças de conceito}) ao longo do tempo.
        \item<4 -> A ocorrência de \alert{mudanças de conceito} (\textit{concept drifts}) pode impactar a acurácia da técnica aplicada.
      \end{itemize}
\end{frame}

\begin{frame}{Introdução}
    \begin{itemize}
        \item<1 -> A atualização periódica de modelos, apesar de computacionalmente ineficiente, foi utilizada como estratégia para mitigar a perda de acurácia causada por tais mudanças.
        \item<2 -> Visando obter soluções computacionalmente eficientes e com maior precisão, pesquisadores propuseram novos métodos de detecção de mudança de conceito baseados em monitoramento.
      \end{itemize}
\end{frame}


\begin{frame}{Introdução}
    \begin{itemize}
        \item<1 -> Entretanto, os métodos disponíveis na literatura ainda apresentam limitações ao serem aplicados em cenários com FCDs \cite{Aggarwal:2006:DSM:1196418}:
        \begin{itemize}
            \item<2 -> Necessidade de rotulação;
            \item<2 -> Eficiência computacional (tempo de resposta e uso de recursos).
        \end{itemize}
      \end{itemize}
\end{frame}

\begin{frame}{Introdução}
    \begin{itemize}
        \item<1 -> Visando superar essas limitações, este trabalho propõe um novo método de detecção de mudanças de conceito baseado em \alert{Redes de Função de Base Radial (redes RBF) e Cadeias de Markov};
        \item<2 -> O método proposto se diferencia por detectar mudanças em tempo de execução, de forma computacionalmente eficiente e independente de rótulos.
    \end{itemize}
\end{frame}

\section{Fundamentação Teórica}

\begin{frame}{Lorem Ipsum}
    \begin{itemize}
        \item<1 -> A:
        \begin{itemize}
            \item<2 -> A1;
            \item<2 -> A2.
        \end{itemize}
        \item<3 -> B
      \end{itemize}
\end{frame}

\section{RBFChain}

\begin{frame}{Lorem Ipsum}
    \begin{itemize}
        \item<1 -> A:
        \begin{itemize}
            \item<2 -> A1;
            \item<2 -> A2.
        \end{itemize}
        \item<3 -> B
      \end{itemize}
\end{frame}

\section{Experimentos}

\begin{frame}{Lorem Ipsum}
    \begin{itemize}
        \item<1 -> A:
        \begin{itemize}
            \item<2 -> A1;
            \item<2 -> A2.
        \end{itemize}
        \item<3 -> B
      \end{itemize}
\end{frame}

\section{Conclusões e Trabalhos Futuros}

\begin{frame}{Lorem Ipsum}
    \begin{itemize}
        \item<1 -> A:
        \begin{itemize}
            \item<2 -> A1;
            \item<2 -> A2.
        \end{itemize}
        \item<3 -> B
      \end{itemize}
\end{frame}

\begin{frame}[allowframebreaks]{Referências}

  \bibliography{slides}
  \bibliographystyle{abbrv}

\end{frame}

\end{document}
