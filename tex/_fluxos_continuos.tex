
Fluxos Contínuos de Dados (FCDs) podem ser definidos como sequências ininterruptas e potencialmente infinitas de eventos \cite{Aggarwal:2006:DSM:1196418}.
%
Nestes fluxos, os eventos ocorrem no tempo em uma taxa que não permite o armazenamento permanente em memória.
%
Além disso, por serem de tamanho potencialmente ilimitado, não é possível aplicar os métodos tradicionais de aprendizado em lote, pois existem severas restrições de memória e tempo de processamento \cite{Gama:2007:LDS:1349782}.

As características dos fluxos contínuos de dados implicam nas seguintes restrições aos algoritmos que os processam \cite{bifet2009data}:
%
\begin{enumerate}
    \item É impossível armazenar todos os dados do fluxo. Somente uma pequena parcela pode ser processada e armazenada, enquanto o restante é descartado;
    \item A velocidade de chegada dos eventos no fluxo exige que cada elemento seja processado em tempo de execução;
    \item A distribuição dos dados pode mudar com o tempo. Assim, os dados do passado podem se tornar irrelevantes ou mesmo prejudiciais para a descrição dos conceitos atuais.
\end{enumerate}

A primeira restrição limita a quantidade de memória que os algoritmos que lidam com fluxos contínuos de dados podem utilizar.
%
A segunda, restringe o tempo de processamento para cada evento.
%
Por fim, a terceira torna necessário utilizar técnicas de detecção de mudanças de conceito e métodos de atualização dos dados.
%

